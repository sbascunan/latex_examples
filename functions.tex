\documentclass{article}
	\textheight = 21cm
	\textwidth = 18cm
	\topmargin = -1.5cm
	\oddsidemargin = -1cm
	\parindent = 0mm
\usepackage{amsmath,amssymb,amsfonts,latexsym}
\usepackage[T1]{fontenc}
\usepackage[latin1]{inputenc}
\begin{document}
	\section*{1. Funci\'on par}
		Consideremos una funci\'on real de variable real definida por:
			$$f:A\rightarrow B; x \longmapsto y=f(x)$$
		En tanto que, decimos que $f$ es par si para todo elemento de $x$ en $A$ se cumple para todo $y$ en $B$ que:
			$$f(x)=f(-x)$$
		o bien podemos evidenciar que, por esta ocurrencia, cuando una funci\'on es par, presenta simetr\'ia con respecto del eje de las ordenadas, en conclusi\'on, $f$ es par si
			$$\forall x\in A, f(x)=f(-x)$$
	\section*{2. Funci\'on impar} 
		Consideremos una funci\'on similar a la de antes, definida como sigue:
			$$f:A\rightarrow B; x \longmapsto y=f(x)$$
		Decimos que $f$ es impar si para todo elemento $x$ en $A$ se cumple para todo $y$ en $B$ que:
			$$f(x)=-f(-x)$$
		es decir, $f$ es impar si
			$$\forall x\in A, f(x)=-f(-x)$$
		Como podemos notar, a diferencia del caso anterior en donde la funci\'on es sim\'etrica con respecto del eje de las ordenadas, en este caso las im\'agenes de la funci\'on son sim\'etricas con respecto del origen.
	\section*{3. Inyectividad de una funci\'on}
		Sobre una funci\'on con las mismas caracter\'isticas que las funciones definidas anteriormente, definimos la inyectividad de una funci\'on cuando esta cumple con:
			$$\forall x\in A, x_{1}=x_{2} \Longleftrightarrow f(x_{1})=f(x_{2})$$
		Desde la primera definici�n, podemos extraer una segunda, de donde, si las im\'agenes de $x$ por $f$ son distintas, entonces sus preim\'agenes son distintas:
			$$\forall x\in A, x_{1}\neq x_{2} \Longleftrightarrow f(x_{1})\neq f(x_{2})$$
		Al momento de estudiar la inyectividad de una funci\'on, es conveniente usar la primera definici\'on. Para comprender el concepto, m\'as f\'acil es interpretar la segunda definici\'on.
			
\end{document}